\documentclass[10pt,twocolumn,letterpaper]{article}

\usepackage{cvpr}
\usepackage{times}
\usepackage{epsfig}
\usepackage{graphicx}
\usepackage{amsmath}
\usepackage{amssymb}

% Include other packages here, before hyperref.

% If you comment hyperref and then uncomment it, you should delete
% egpaper.aux before re-running latex.  (Or just hit 'q' on the first latex
% run, let it finish, and you should be clear).
\usepackage[breaklinks=true,bookmarks=false]{hyperref}

\cvprfinalcopy % *** Uncomment this line for the final submission

\def\cvprPaperID{****} % *** Enter the CVPR Paper ID here
\def\httilde{\mbox{\tt\raisebox{-.5ex}{\symbol{126}}}}

% Pages are numbered in submission mode, and unnumbered in camera-ready
%\ifcvprfinal\pagestyle{empty}\fi
\setcounter{page}{1}
\begin{document}

%%%%%%%%% TITLE
\title{CSE/ECE 343: Machine Learning Project Proposal  }

\author{Naman Jindal\\
naman22311@iiitd.ac.in\\
\\
{}
% For a paper whose authors are all at the same institution,
% omit the following lines up until the closing ``}''.
% Additional authors and addresses can be added with ``\and'',
% just like the second author.
% To save space, use either the email address or home page, not both
\and
Nishchay Sharma \\
nishchay22331@iiitd.ac.in\\
\\
{}
\and
Harshil Handoo\\
harshil22206@iiitd.ac.in\\
\\
{}
\and
 \\
himanshu kumar\\
himanshu22215@iiitd.ac.in\\
{}
}


\maketitle
%\thispagestyle{empty}

%%%%%%%%% 
\begin{abstract}
   
\end{abstract}

%%%%%%%%% BODY TEXT
\section{Motivation}

Please follow the steps outlined below when submitting your manuscript to
the IEEE Computer Society Press.  This style guide now has several
important modifications (for example, you are no longer warned against the
use of sticky tape to attach your artwork to the paper), so all authors
should read this new version.

%-------------------------------------------------------------------------
\subsection{Language}

All manuscripts must be in English.

\subsection{Dual submission}

Please refer to the author guidelines on the CVPR 2020 web page for a
discussion of the policy on dual submissions.

\subsection{Paper length}
Papers, excluding the references section,
must be no longer than eight pages in length. The references section
will not be included in the page count, and there is no limit on the
length of the references section. For example, a paper of eight pages
with two pages of references would have a total length of 10 pages.
{\bf There will be no extra page charges for CVPR 2020.}

Overlength papers will simply not be reviewed.  This includes papers
where the margins and formatting are deemed to have been significantly
altered from those laid down by this style guide.  Note that this
\LaTeX\ guide already sets figure captions and references in a smaller font.
The reason such papers will not be reviewed is that there is no provision for
supervised revisions of manuscripts.  The reviewing process cannot determine
the suitability of the paper for presentation in eight pages if it is
reviewed in eleven.  

%-------------------------------------------------------------------------
\subsection{The ruler}
The \LaTeX\ style defines a printed ruler which should be present in the
version submitted for review.  The ruler is provided in order that
reviewers may comment on particular lines in the paper without
circumlocution.  If you are preparing a document using a non-\LaTeX\
document preparation system, please arrange for an equivalent ruler to
appear on the final output pages.  The presence or absence of the ruler
should not change the appearance of any other content on the page.  The
camera ready copy should not contain a ruler. (\LaTeX\ users may uncomment
the \verb'\cvprfinalcopy' command in the document preamble.)  Reviewers:
note that the ruler measurements do not align well with lines in the paper
--- this turns out to be very difficult to do well when the paper contains
many figures and equations, and, when done, looks ugly.  Just use fractional
references (e.g.\ this line is $095.5$), although in most cases one would
expect that the approximate location will be adequate.

\subsection{Expectation}

The goal is to develop a robust machine learning model capable of accurately predicting crop yields based on various input factors such as weather conditions, soil properties, and crop type. This model could be used as a decision-support tool for farmers, helping them optimize their resources and improve crop yield. The project also aims to contribute to the field of precision agriculture by providing insights that can lead to more sustainable farming practices.


\end{document}
5